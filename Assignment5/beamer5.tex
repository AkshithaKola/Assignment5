\documentclass{beamer}
\usetheme{CambridgeUS}

\title{Assignment 5 : Papoulis Textbook }
\author{Akshitha Kola}
\date{\today}
\logo{\large \LaTeX{}}

\usepackage{amsmath}
\setbeamertemplate{caption}[numbered]{}
\providecommand{\pr}[1]{\ensuremath{\Pr\left(#1\right)}}
\providecommand{\cbrak}[1]{\ensuremath{\left\{#1\right\}}}

\begin{document}

\begin{frame}
    \titlepage 
\end{frame}

\logo{}

\begin{frame}{Outline}
    \tableofcontents
\end{frame}

\section{Question}
\begin{frame}{Question}
    \begin{block}{Chapter 3 example 3.1}
    The cartesian product of the sets SI = $\{car, apple, bird\}$ S2 = $\{h, t\}$ is?
    \end{block}
\end{frame}

\section{Solution}
\begin{frame}{Solution}
\frametitle{Solution}
Given SI = $\{car, apple, bird\}$ and S2 = $\{h, t\}$ \\
Let the cartesian product of the sets S1 and S2 be S.\\
Then S = S1 x S2 \\
$\implies$ S = $\{(car,h),(car,t),(apple,h),(apple,t),(bird,h),(bird,t)\}$
\end{frame}

\end{document}